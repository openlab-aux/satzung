%!TEX TS-program = xelatex
%!TEX encoding = UTF-8 Unicode

% Druck
%\documentclass[a5paper, ngerman, twoside, 10pt]{scrreprt}

% Screen
\documentclass[a5paper, ngerman, 10pt]{scrreprt}

\usepackage[a5paper]{geometry}
\usepackage[ngerman]{babel}

% Diese Pakete lassen sich nur mit XeLaTex oder LuaLaTex verwenden
\usepackage{fontspec}
\usepackage{xunicode}
\defaultfontfeatures{Mapping=tex-text} % To support LaTeX quoting style

% Optischer Randausgleich
\usepackage[protrusion=true]{microtype}

\usepackage{enumerate}

\usepackage[hyphens]{url}

% Generiert ein PDF-Inhaltsverzeichnis und klickbare Links
\usepackage[hidelinks]{hyperref}
\urlstyle{same} % Keine Monotype-Schriftart für URLs

% OpenLab Farben RGB
\usepackage{xcolor}
\definecolor{olagrau}{HTML}{4d4d4d}
\definecolor{olablau}{HTML}{0088aa}

% OpenLab-Schriftarten (Müssen als Systemfonts vorhanden sein)
\newfontfamily\PHead{Ubuntu}
%\setmainfont{DejaWeb}
\setmainfont{DejaVu Sans}

% Zeilenabstand 1.5
\usepackage[onehalfspacing]{setspace}

% Neudefinition der Schriften
\setkomafont{chapter}{\linespread{0.9}\Huge\color{olablau}\PHead}
\setkomafont{section}{\linespread{0.9}\LARGE\color{olagrau}\PHead}
\setkomafont{subsection}{\linespread{0.9}\Large\color{olagrau}\PHead}
\setkomafont{subsubsection}{\linespread{0.9}\large\color{olagrau}\PHead}
\setkomafont{chapterentry}{\large\color{olagrau}\PHead}
\setkomafont{pagenumber}{\small\color{olablau}}

\usepackage{pdfpages}

% Schusterjungen und Hurenkinder vermeiden
\usepackage[defaultlines=3,all]{nowidow}

% Paragraphenzeichen vor der Nummerierung
\renewcommand*\thesection{\S~\arabic{section}}

\begin{document}
\includepdf[fitpaper]{deckblatt.pdf}

\thispagestyle{empty}
\vspace*{\fill}
\begin{footnotesize}
    \begin{singlespace}
        \noindent Satzung des OpenLab Augsburg e.~V.\\
        Elisenstraße 1\\
        86159 Augsburg\\
        \\
        Beschlossen am 23.03.2013 auf der\\
        Mitgliederversammlung in Augsburg\\
        \\
        2. Auflage 2015, Augsburg\\
        \\
        \url{https://openlab-augsburg.de}
    \end{singlespace}
\end{footnotesize}
\clearpage

\tableofcontents
\clearpage


\section*{Präambel}
\addcontentsline{toc}{section}{Präambel}
Das OpenLab schafft einen Rahmen für Projekte, die praktisches,
interkulturelles und kunsthandwerkliches Tun fördern und allen Menschen
selbstbestimmte, selbstermächtigende Bildung ermöglichen. Es unterstützt
schöpferischen Pioniergeist und kritische Technikbegeisterung im Sinne einer
Demokratisierung von Produktionswissen.


\section{Name, Sitz, Eintragung, Geschäftsjahr}
\begin{enumerate}[(1)]
    \item Der Verein trägt den Namen OpenLab (e.V.).
    \item Er hat den Sitz in Augsburg.
    \item Er verfolgt ausschließlich und unmittelbar gemeinnützige Zwecke im
        Sinne des Abschnitts „Steuerbegünstigte Zwecke „der Abgabenordnung.
    \item Er soll in das Vereinsregister eingetragen werden.
    \item Geschäftsjahr ist das Kalenderjahr.
\end{enumerate}


\section{Vereinszweck}
\begin{enumerate}[(1)]
    \item Insbesondere führt der Verein Vorhaben der Forschung, Wissenschaft
        und Volks- und Berufsbildung, sowie der Kunst und Kultur und
        internationaler Gesinnung durch, oder fördert und unterstützt diese.
    \item Der Vereinszweck soll unter anderem durch folgende Mittel erreicht
        werden:
        \begin{itemize}
            \item Veranstaltung von Schulungen und Workshops zur Aus- und
                Weiterbildung sowie in Kunsttechniken und allgemeinen
                Fertigungsverfahren inklusive der zugehörigen
                Werkstoffkunde
            \item Veranstaltung von Vorträgen, Seminaren und Tagungen, auch und
                insbesondere zur Behandlung von offenen Fragen und aktuellen
                Entwicklungen in o.g. Themenbereichen
            \item Vernetzung mit bestehenden lokalen, regionalen und
                internationalen Gruppen, z.~b.  User-Groups, Stammtische,
                Computerclubs, CoworkingSpaces, offenen
                Werkstattgemeinschaften, Künstlergruppen etc. sowie
                Durchführung von nationalen und internationalen Kongressen und
                Konferenzen
            \item die Entwicklung, Erprobung und Etablierung von Konzepten und
                Angeboten zur kunsthandwerklichen / technischen /
                künstlerischen / sozialen Bildung jenseits konventioneller
                Berufsausbildung im Sinne der Selbstbefähigung von Menschen,
                ihr Lebensumfeld, wie auch Dinge des täglichen Bedarfs oder von
                Interesse in Eigenarbeit und in Eigenregie zu erschaffen oder
                instand zu halten
            \item die Förderung von Angeboten, die zur Bewahrung und Entfaltung
                kunsthandwerklicher, kultureller und sozialer Fähigkeiten
                dienen und die Weitergabe von Wissen und Fertigkeiten an
                Menschen ungeachtet ihres Alters, Herkunft, Geschlechts oder
                kultureller Orientierung im Sinne gemeinschaftlicher und
                gegenseitiger Unterstützung zu selbstbestimmter Bildung,
                fördern. (Hilfe zur Selbsthilfe)
        \end{itemize}
\end{enumerate}


\section{Selbstlosigkeit}
Die Körperschaft ist selbstlos tätig; sie verfolgt nicht in erster Linie
eigenwirtschaftliche Zwecke. Mittel der Körperschaft dürfen nur für die
satzungsmäßigen Zwecke verwendet werden. Die Mitglieder erhalten in ihrer
Eigenschaft als Mitglieder keine Zuwendungen aus Mitteln der Körperschaft. Es
darf keine Person durch Ausgaben, die dem Zweck der Körperschaft fremd sind,
oder durch unverhältnismäßig hohe Vergütungen begünstigt werden.


\section{Mitgliedschaft}
\begin{enumerate}[(1)]
    \item Mitglied des OpenLab e.V. kann jede juristische und natürliche Person
        werden, die seine Ziele unterstützt
    \item Über den Antrag auf Aufnahme in den OpenLab e.V. entscheidet der
        Vorstand.
    \item Die Mitgliedschaft endet durch Austritt, Ausschluss, Auflösung oder
        Tod
    \item Der Austritt eines Mitgliedes ist jederzeit möglich. Er erfolgt durch
        schriftliche Erklärung gegenüber einem Mitglied des Vorstands.
    \item Wenn ein Mitglied gegen die Ziele und Interessen des Vereins schwer
        verstoßen hat oder trotz Mahnung mit dem Beitrag für mindestens 1 Jahr
        im Rückstand bleibt, so kann es durch den Vorstand mit sofortiger
        Wirkung ausgeschlossen werden. Dem Mitglied muss vor der
        Beschlussfassung Gelegenheit zur Rechtfertigung bzw. Stellungnahme
        gegeben werden. Gegen den Ausschließungsbeschluss kann innerhalb einer
        Frist von drei Monaten nach Mitteilung des Ausschlusses Berufung
        eingelegt werden, über den die nächste Mitgliederversammlung
        entscheidet.
    \item Es wird unterschieden zwischen aktiven Mitgliedern und
        Fördermitgliedschaften.  Fördermitglieder sind juristische oder
        natürliche Personen, die lediglich passiv fördern (z.B. Finanziell)
        oder juristische Personen, die nicht als gemeinnützig anerkannt sind.
        Mitgliedsbeiträge werden erhoben.
\end{enumerate}


\section{Beiträge}
Die Mitglieder zahlen Beiträge nach Maßgabe einer Gebührenordnung. Die
Gebührenordnung wird durch Beschluss der Mitgliederversammlung verabschiedet.
Zur Festlegung der Beitragshöhe und -fälligkeit ist eine einfache Mehrheit der
in der Mitgliederversammlung anwesenden, stimmberechtigten Vereinsmitglieder
erforderlich.


\section{Organe des Vereins}
Organe des Vereins sind
\begin{enumerate}[a.]
    \item der Vorstand
    \item die Mitgliederversammlung
\end{enumerate}


\section{Der Vorstand}
\begin{enumerate}[(1)]
    \item Der Vorstand besteht aus drei Mitgliedern. Er vertritt den Verein
        gerichtlich und außergerichtlich. Je zwei Vorstandsmitglieder sind
        gemeinsam vertretungsberechtigt.
    \item Der Vorstand wird von der Mitgliederversammlung für die Dauer von 2
        Jahren gewählt. Die Wiederwahl der Vorstandsmitglieder ist bis zu zwei
        Mal hintereinander möglich. Die jeweils amtierenden Vorstandsmitglieder
        bleiben nach Ablauf ihrer Amtszeit im Amt, bis Nachfolger gewählt sind.
    \item Dem Vorstand obliegt die Führung der laufenden Geschäfte des Vereins.
    \item Vorstandssitzungen finden jährlich mindestens zweimal statt. Die
        Einladung zu Vorstandssitzungen erfolgt schriftlich oder elektronisch.
        Vorstandssitzungen sind beschlussfähig, wenn mindestens die Hälfte der
        Vorstandsmitglieder persönlich anwesend sind. Die Stimmen sind nicht
        übertragbar.
    \item Der Vorstand fasst seine Beschlüsse mit einfacher Mehrheit. Bei
        Stimmengleichheit gilt der Antrag als abgelehnt.
    \item Beschlüsse des Vorstands können bei Eilbedürftigkeit auch schriftlich
        oder fernmündlich gefasst werden, wenn alle Vorstandsmitglieder ihre
        Zustimmung zu diesem Verfahren schriftlich oder fernmündlich erklären.
        Schriftlich oder fernmündlich gefasste Vorstandsbeschlüsse sind
        schriftlich niederzulegen und von mindestens einem Vorstandsmitglied zu
        unterzeichnen.
\end{enumerate}


\section{Mitgliederversammlung}
\begin{enumerate}[(1)]
    \item Die Mitgliederversammlung ist einmal jährlich einzuberufen.
    \item Eine außerordentliche Mitgliederversammlung ist einzuberufen, wenn es
        das Vereinsinteresse erfordert oder wenn die Einberufung von 10% der
        Vereinsmitglieder schriftlich und unter Angabe des Zweckes und der
        Gründe verlangt wird.
    \item Die Einberufung der regulären Mitgliederversammlung erfolgt
        schriftlich (Post/Email) durch den Vorstand unter Wahrung einer
        Einladungsfrist von mindestens 4 Wochen bei gleichzeitiger Bekanntgabe
        der Tagesordnung. Bei einer außerordentlichen Mitgliederversammlung
        kann die Frist auf 2 Wochen verkürzt werden. Die Frist beginnt mit dem
        auf die Absendung des Einladungsschreibens folgenden Tag. Es gilt das
        Datum des Poststempels/der abgesendeten Email. Das Einladungsschreiben
        gilt dem Mitglied als zugegangen, wenn es an die letzte vom Mitglied
        des Vereins schriftlich bekannt gegebene Adresse/Email gerichtet ist.
    \item Die Mitgliederversammlung als das oberste beschlussfassende
        Vereinsorgan ist grundsätzlich für alle Aufgaben zuständig, sofern
        bestimmte Aufgaben gemäß dieser Satzung nicht einem anderen
        Vereinsorgan übertragen wurden. Ihr sind insbesondere die
        Jahresrechnung und der Jahresbericht zur Beschlussfassung über die
        Genehmigung und die Entlastung des Vorstandes schriftlich vorzulegen.
        Sie bestellt zwei Rechnungsprüfer, die weder dem Vorstand noch einem
        vom Vorstand berufenen Gremium angehören dürfen, um die Buchführung
        einschließlich Jahresabschluss zu prüfen und über das Ergebnis vor der
        Mitgliederversammlung zu berichten. Die Mitgliederversammlung
        entscheidet z. B. auch über:
    \begin{enumerate}[a.]
        \item Gebührenbefreiungen
        \item Aufgaben des Vereins
        \item An- und Verkauf sowie Belastung von Grundbesitz
        \item Beteiligung an Gesellschaften
        \item Aufnahme von Darlehen
        \item Genehmigung aller Geschäftsordnungen für den Vereinsbereich
        \item Mitgliedsbeiträge
        \item Satzungsänderungen
        \item Auflösung des Vereins
    \end{enumerate}
    \item Jede satzungsmäßig einberufene Mitgliederversammlung wird als
        beschlussfähig anerkannt ohne Rücksicht auf die Zahl der erschienenen
        Vereinsmitglieder. Jedes aktive Mitglied hat eine Stimme.
        Fördermitglieder haben lediglich beratende Stimme. Das Stimmrecht kann
        nur persönlich anwesend wahrgenommen werden.
    \item Die Mitgliederversammlung fasst ihre Beschlüsse mit einfacher
        Mehrheit.  Bei Stimmengleichheit gilt ein Antrag als abgelehnt.
\end{enumerate}


\section{Satzungsänderung}
\begin{enumerate}[(1)]
    \item Für Satzungsänderungen ist eine 2/3-Mehrheit der erschienenen
        Vereinsmitglieder erforderlich.
    \item Satzungsänderungen, die von Aufsichts-, Gerichts- oder Finanzbehörden
        aus formalen Gründen verlangt werden, kann der Vorstand von sich aus
        vornehmen. Diese Satzungsänderungen müssen allen Vereinsmitgliedern
        alsbald schriftlich mitgeteilt werden.
\end{enumerate}


\section{Beurkundung von Beschlüssen}
Die in Vorstandssitzungen und in Mitgliederversammlungen gefassten Beschlüsse
sind schriftlich niederzulegen und vom Vorstand zu unterzeichnen.


\section{Auflösung des Vereins und Vermögensbindung}
\begin{enumerate}[(1)]
    \item Für den Beschluss, den Verein aufzulösen, ist eine 3/4-Mehrheit der
        in der Mitgliederversammlung anwesenden Mitglieder erforderlich. Der
        Beschluss kann nur nach rechtzeitiger Ankündigung in der Einladung zur
        Mitgliederversammlung gefasst werden.
    \item Bei Auflösung oder Aufhebung des Vereins oder bei Wegfall
        steuerbegünstigter Zwecke fällt das Vermögen des Vereins dem
        gemeinnützigen Verein „Verbund Offener Werkstätten e.V.“ mit Sitz in
        Augsburg (VR31850B).
\end{enumerate}

\vspace{\fill}
Augsburg, 26. März 2013

\end{document}
