%!TEX TS-program = xelatex
%!TEX encoding = UTF-8 Unicode

% Druck
%\documentclass[a5paper, ngerman, twoside, 10pt]{scrreprt}

% Screen
\documentclass[a5paper, ngerman, 10pt]{scrreprt}

\usepackage[a5paper]{geometry}
\usepackage[ngerman]{babel}

% to change itemization style
\usepackage{paralist}

\usepackage[hyphens]{url}

% OpenLab RGB colors
\usepackage{xcolor}
\definecolor{olagrau}{HTML}{4d4d4d}
\definecolor{olablau}{HTML}{0088aa}

% set line spacing to 1.5
\usepackage[onehalfspacing]{setspace}

\usepackage{pdfpages}

% XeLaTex/LuaLaTex only
\ifx\csname XeTeXrevision\endcsname
    \usepackage{fontspec}
    \usepackage{xunicode}
    \defaultfontfeatures{Mapping=tex-text} % to support LaTeX quoting style

    % generates a PDF table of contents with clickable links
    \usepackage[hidelinks]{hyperref}
    \urlstyle{same} % don't use monotype font for URLs

    % OpenLab fonts (have to be available as system fonts)
    \newfontfamily\PHead{Ubuntu}
    %\setmainfont{DejaWeb}
    \setmainfont{DejaVu Sans}

    % avoid widows and orphans
    \usepackage[defaultlines=3,all]{nowidow}

    % redeclare fonts
    \setkomafont{chapter}{\linespread{0.9}\Huge\color{olablau}\PHead}
    \setkomafont{section}{\linespread{0.9}\LARGE\color{olagrau}\PHead}
    \setkomafont{subsection}{\linespread{0.9}\Large\color{olagrau}\PHead}
    \setkomafont{subsubsection}{\linespread{0.9}\large\color{olagrau}\PHead}
    \setkomafont{chapterentry}{\large\color{olagrau}\PHead}
    \setkomafont{pagenumber}{\small\color{olablau}}
\fi

% prepend section signs to items
\renewcommand*\thesection{\S~\arabic{section}}

\begin{document}
\includepdf[fitpaper]{deckblatt.pdf}

\thispagestyle{empty}
\vspace*{\fill}
\begin{footnotesize}
    \begin{singlespace}
        \noindent Satzung des OpenLab Augsburg e.~V.\\
        Elisenstra"se 1\\
        86159 Augsburg\\
        \\
        Beschlossen am 23.03.2013 auf der\\
        Mitgliederversammlung in Augsburg\\
        \\
        2. Auflage 2015, Augsburg\\
        \\
        \url{https://openlab-augsburg.de}
    \end{singlespace}
\end{footnotesize}
\clearpage

\tableofcontents
\clearpage


\section*{Pr"aambel}
\addcontentsline{toc}{section}{Pr"aambel}
Das OpenLab schafft einen Rahmen f"ur Projekte, die praktisches,
interkulturelles und kunsthandwerkliches Tun f"ordern und allen Menschen
selbstbestimmte, selbsterm"achtigende Bildung erm"oglichen. Es unterst"utzt
sch"opferischen Pioniergeist und kritische Technikbegeisterung im Sinne einer
Demokratisierung von Produktionswissen.


\section{Name, Sitz, Eintragung, Gesch"aftsjahr}
\begin{compactenum}[(1)]
    \item Der Verein tr"agt den Namen OpenLab (e.V.).
    \item Er hat den Sitz in Augsburg.
    \item Er verfolgt ausschlie"slich und unmittelbar gemeinn"utzige Zwecke im
        Sinne des Abschnitts "`Steuerbeg"unstigte Zwecke"' der Abgabenordnung.
    \item Er soll in das Vereinsregister eingetragen werden.
    \item Gesch"aftsjahr ist das Kalenderjahr.
\end{compactenum}


\section{Vereinszweck}
\begin{compactenum}[(1)]
    \item Insbesondere f"uhrt der Verein Vorhaben der Forschung, Wissenschaft
        und Volks- und Berufsbildung, sowie der Kunst und Kultur und
        internationaler Gesinnung durch, oder f"ordert und unterst"utzt diese.
    \item Der Vereinszweck soll unter anderem durch folgende Mittel erreicht
        werden:
        \begin{itemize}
            \item Veranstaltung von Schulungen und Workshops zur Aus- und
                Weiterbildung sowie in Kunsttechniken und allgemeinen
                Fertigungsverfahren inklusive der zugeh"origen Werkstoffkunde
            \item Veranstaltung von Vortr"agen, Seminaren und Tagungen, auch und
                insbesondere zur Behandlung von offenen Fragen und aktuellen
                Entwicklungen in o.g. Themenbereichen
            \item Vernetzung mit bestehenden lokalen, regionalen und
                internationalen Gruppen, z.~b.  User-Groups, Stammtische,
                Computerclubs, CoworkingSpaces, offenen Werkstattgemeinschaften,
                K"unstlergruppen etc. sowie Durchf"uhrung von nationalen und
                internationalen Kongressen und Konferenzen
            \item die Entwicklung, Erprobung und Etablierung von Konzepten und
                Angeboten zur kunsthandwerklichen / technischen /
                k"unstlerischen / sozialen Bildung jenseits konventioneller
                Berufsausbildung im Sinne der Selbstbef"ahigung von Menschen,
                ihr Lebensumfeld, wie auch Dinge des t"aglichen Bedarfs oder von
                Interesse in Eigenarbeit und in Eigenregie zu erschaffen oder
                instand zu halten
            \item die F"orderung von Angeboten, die zur Bewahrung und Entfaltung
                kunsthandwerklicher, kultureller und sozialer F"ahigkeiten
                dienen und die Weitergabe von Wissen und Fertigkeiten an
                Menschen ungeachtet ihres Alters, Herkunft, Geschlechts oder
                kultureller Orientierung im Sinne gemeinschaftlicher und
                gegenseitiger Unterst"utzung zu selbstbestimmter Bildung,
                f"ordern. (Hilfe zur Selbsthilfe)
        \end{itemize}
\end{compactenum}


\section{Selbstlosigkeit}
Die K"orperschaft ist selbstlos t"atig; sie verfolgt nicht in erster Linie
eigenwirtschaftliche Zwecke. Mittel der K"orperschaft d"urfen nur f"ur die
satzungsm"a"sigen Zwecke verwendet werden. Die Mitglieder erhalten in ihrer
Eigenschaft als Mitglieder keine Zuwendungen aus Mitteln der K"orperschaft. Es
darf keine Person durch Ausgaben, die dem Zweck der K"orperschaft fremd sind,
oder durch unverh"altnism"a"sig hohe Verg"utungen beg"unstigt werden.


\section{Mitgliedschaft}
\begin{compactenum}[(1)]
    \item Mitglied des OpenLab e.V. kann jede juristische und nat"urliche Person
        werden, die seine Ziele unterst"utzt
    \item "Uber den Antrag auf Aufnahme in den OpenLab e.V. entscheidet der
        Vorstand.
    \item Die Mitgliedschaft endet durch Austritt, Ausschluss, Aufl"osung oder
        Tod
    \item Der Austritt eines Mitgliedes ist jederzeit m"oglich. Er erfolgt durch
        schriftliche Erkl"arung gegen"uber einem Mitglied des Vorstands.
    \item Wenn ein Mitglied gegen die Ziele und Interessen des Vereins schwer
        versto"sen hat oder trotz Mahnung mit dem Beitrag f"ur mindestens 1 Jahr
        im R"uckstand bleibt, so kann es durch den Vorstand mit sofortiger
        Wirkung ausgeschlossen werden. Dem Mitglied muss vor der
        Beschlussfassung Gelegenheit zur Rechtfertigung bzw. Stellungnahme
        gegeben werden. Gegen den Ausschlie"sungsbeschluss kann innerhalb einer
        Frist von drei Monaten nach Mitteilung des Ausschlusses Berufung
        eingelegt werden, "uber den die n"achste Mitgliederversammlung
        entscheidet.
    \item Es wird unterschieden zwischen aktiven Mitgliedern und
        F"ordermitgliedschaften.  F"ordermitglieder sind juristische oder
        nat"urliche Personen, die lediglich passiv f"ordern (z.B. Finanziell)
        oder juristische Personen, die nicht als gemeinn"utzig anerkannt sind.
        Mitgliedsbeitr"age werden erhoben.
\end{compactenum}


\section{Beitr"age}
Die Mitglieder zahlen Beitr"age nach Ma"sgabe einer Geb"uhrenordnung. Die
Geb"uhrenordnung wird durch Beschluss der Mitgliederversammlung verabschiedet.
Zur Festlegung der Beitragsh"ohe und -f"alligkeit ist eine einfache Mehrheit der
in der Mitgliederversammlung anwesenden, stimmberechtigten Vereinsmitglieder
erforderlich.


\section{Organe des Vereins}
Organe des Vereins sind
\begin{compactenum}[a.]
    \item der Vorstand
    \item die Mitgliederversammlung
\end{compactenum}


\section{Der Vorstand}
\begin{compactenum}[(1)]
    \item Der Vorstand besteht aus drei Mitgliedern. Er vertritt den Verein
        gerichtlich und au"sergerichtlich. Je zwei Vorstandsmitglieder sind
        gemeinsam vertretungsberechtigt.
    \item Der Vorstand wird von der Mitgliederversammlung f"ur die Dauer von 2
        Jahren gew"ahlt. Die Wiederwahl der Vorstandsmitglieder ist bis zu zwei
        Mal hintereinander m"oglich. Die jeweils amtierenden Vorstandsmitglieder
        bleiben nach Ablauf ihrer Amtszeit im Amt, bis Nachfolger gew"ahlt sind.
    \item Dem Vorstand obliegt die F"uhrung der laufenden Gesch"afte des
        Vereins.
    \item Vorstandssitzungen finden j"ahrlich mindestens zweimal statt. Die
        Einladung zu Vorstandssitzungen erfolgt schriftlich oder elektronisch.
        Vorstandssitzungen sind beschlussf"ahig, wenn mindestens die H"alfte der
        Vorstandsmitglieder pers"onlich anwesend sind. Die Stimmen sind nicht
        "ubertragbar.
    \item Der Vorstand fasst seine Beschl"usse mit einfacher Mehrheit. Bei
        Stimmengleichheit gilt der Antrag als abgelehnt.
    \item Beschl"usse des Vorstands k"onnen bei Eilbed"urftigkeit auch
        schriftlich oder fernm"undlich gefasst werden, wenn alle
        Vorstandsmitglieder ihre Zustimmung zu diesem Verfahren schriftlich oder
        fernm"undlich erkl"aren.  Schriftlich oder fernm"undlich gefasste
        Vorstandsbeschl"usse sind schriftlich niederzulegen und von mindestens
        einem Vorstandsmitglied zu unterzeichnen.
\end{compactenum}


\section{Mitgliederversammlung}
\begin{compactenum}[(1)]
    \item Die Mitgliederversammlung ist einmal j"ahrlich einzuberufen.
    \item Eine au"serordentliche Mitgliederversammlung ist einzuberufen, wenn es
        das Vereinsinteresse erfordert oder wenn die Einberufung von 10\% der
        Vereinsmitglieder schriftlich und unter Angabe des Zweckes und der
        Gr"unde verlangt wird.
    \item Die Einberufung der regul"aren Mitgliederversammlung erfolgt
        schriftlich (Post/Email) durch den Vorstand unter Wahrung einer
        Einladungsfrist von mindestens 4 Wochen bei gleichzeitiger Bekanntgabe
        der Tagesordnung. Bei einer au"serordentlichen Mitgliederversammlung
        kann die Frist auf 2 Wochen verk"urzt werden. Die Frist beginnt mit dem
        auf die Absendung des Einladungsschreibens folgenden Tag. Es gilt das
        Datum des Poststempels/der abgesendeten Email. Das Einladungsschreiben
        gilt dem Mitglied als zugegangen, wenn es an die letzte vom Mitglied des
        Vereins schriftlich bekannt gegebene Adresse/Email gerichtet ist.
    \item Die Mitgliederversammlung als das oberste beschlussfassende
        Vereinsorgan ist grunds"atzlich f"ur alle Aufgaben zust"andig, sofern
        bestimmte Aufgaben gem"a"s dieser Satzung nicht einem anderen
        Vereinsorgan "ubertragen wurden. Ihr sind insbesondere die
        Jahresrechnung und der Jahresbericht zur Beschlussfassung "uber die
        Genehmigung und die Entlastung des Vorstandes schriftlich vorzulegen.
        Sie bestellt zwei Rechnungspr"ufer, die weder dem Vorstand noch einem
        vom Vorstand berufenen Gremium angeh"oren d"urfen, um die Buchf"uhrung
        einschlie"slich Jahresabschluss zu pr"ufen und "uber das Ergebnis vor
        der Mitgliederversammlung zu berichten. Die Mitgliederversammlung
        entscheidet z. B. auch "uber:
    \begin{compactenum}[a.]
        \item Geb"uhrenbefreiungen
        \item Aufgaben des Vereins
        \item An- und Verkauf sowie Belastung von Grundbesitz
        \item Beteiligung an Gesellschaften
        \item Aufnahme von Darlehen
        \item Genehmigung aller Gesch"aftsordnungen f"ur den Vereinsbereich
        \item Mitgliedsbeitr"age
        \item Satzungs"anderungen
        \item Aufl"osung des Vereins
    \end{compactenum}
    \item Jede satzungsm"a"sig einberufene Mitgliederversammlung wird als
        beschlussf"ahig anerkannt ohne R"ucksicht auf die Zahl der erschienenen
        Vereinsmitglieder. Jedes aktive Mitglied hat eine Stimme.
        F"ordermitglieder haben lediglich beratende Stimme. Das Stimmrecht kann
        nur pers"onlich anwesend wahrgenommen werden.
    \item Die Mitgliederversammlung fasst ihre Beschl"usse mit einfacher
        Mehrheit. Bei Stimmengleichheit gilt ein Antrag als abgelehnt.
\end{compactenum}


\section{Satzungs"anderung}
\begin{compactenum}[(1)]
    \item F"ur Satzungs"anderungen ist eine 2/3-Mehrheit der erschienenen
        Vereinsmitglieder erforderlich.
    \item Satzungs"anderungen, die von Aufsichts-, Gerichts- oder
        Finanzbeh"orden aus formalen Gr"unden verlangt werden, kann der Vorstand
        von sich aus vornehmen. Diese Satzungs"anderungen m"ussen allen
        Vereinsmitgliedern alsbald schriftlich mitgeteilt werden.
\end{compactenum}


\section{Beurkundung von Beschl"ussen}
Die in Vorstandssitzungen und in Mitgliederversammlungen gefassten Beschl"usse
sind schriftlich niederzulegen und vom Vorstand zu unterzeichnen.


\section{Aufl"osung des Vereins und Verm"ogensbindung}
\begin{compactenum}[(1)]
    \item F"ur den Beschluss, den Verein aufzul"osen, ist eine 3/4-Mehrheit der
        in der Mitgliederversammlung anwesenden Mitglieder erforderlich. Der
        Beschluss kann nur nach rechtzeitiger Ank"undigung in der Einladung zur
        Mitgliederversammlung gefasst werden.
    \item Bei Aufl"osung oder Aufhebung des Vereins oder bei Wegfall
        steuerbeg"unstigter Zwecke f"allt das Verm"ogen des Vereins dem
        gemeinn"utzigen Verein "`Verbund Offener Werkst"atten e.V."' mit Sitz in
        Augsburg (VR31850B).
\end{compactenum}

\vspace{\fill}
Augsburg, 26. M"arz 2013

\end{document}
